%!Mode::"Tex:UTF-8"
\documentclass{article}
\bibliographystyle{plain}
\usepackage{indentfirst}
\usepackage{picinpar,graphicx}
\usepackage{cite}
\usepackage{amsmath}
\usepackage{amssymb}
\DeclareMathOperator*{\argmax}{argmax}
\setlength{\parindent}{2em}
\author{Hongzhi Liu}
\title{Analysis Step of Matching Method}
\begin{document}
\maketitle
\par	
\section{2D-3D Matching Method}
    In the last article, I give a brief introduction about a global method that harnesses global contextual information exhibited both within the query image and among all the 3D points in the map, which is presented by Professor Li. And I will learn more about key steps of this method.
    
    First of all, They obtain the co-visibility relationship among 3D points by using the database images. And they assume our 3D map was pre-computed via Structure-from-Motion technique using a large set of database images. Secondly, Professor Li detect a set of 2D feature points along with their view-invariant descriptors and find a set of tentative matches from the 3D graph nodes, by comparing their descriptor similarity via an efficient vocabulary-tree search mechanism. Besides, The team seek a global match between 2D query image and 3D map is to run a Random Walk algorithm on this graph. Finally, they try to recover one-to-one correspondences in order to facilitate camera pose computation.

	
	\section{To Verify the Global Search Effective}
	
    In order to evaluate whether or not the use of global contextua information is effective, Professor Li compare their method with the Active Search method which is considered as the state-of-the-art local search methods. They conducted experiments on the metric version of Dubrovnik dataset with sub-maps with reduced sizes of up to 40, 000 map points \cite{Liu2017Efficient}.
    
    After running both algorithms, They compare the histograms of the obtained inlier ratios. The higher the inlier ratio is, the better the method. Figure. 1 gives the distributions of inlier ratios for the two methods. From this, people can clearly see that global method mentioned aboved statistically outperforms Active Search.

\begin{figure}[ht]
		\centering
		\includegraphics[scale=0.5]{1.png}
		\caption{(a). Compare the two histograms of inlier ratios for the 800 query images of Dubrovnik. Red: histogram by our method; Light-blue: histogram by Active-Search. (b). The absolute improvement in terms of inlier numbers over all query images from Dubrovnik. A positive-valued ‘difference’ means more inliers are detected by our method. Our method consistently outperforms the local Active-Search method for almost all 800 queries.}
\end{figure}

\bibliography{1}

\end{document} 